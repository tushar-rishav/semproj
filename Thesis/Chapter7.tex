\section{Summary}
This project investigates the modeling of building heating and cooling load through an intelligent approach that takes into account the
Relative Compactness, Surface Area, Wall Area, Roof Area, Overall Height, Orientation, Glazing Area and Glazing Area Distribution. These findings are particularly compelling given the accurate prediction, and also because we can easily infer the output variables in a matter of few seconds
without requiring the painstaking design of a new building in a simulation tool. 
Further exploration of the statistical relationship between eight input variables and the two output variables was done.

\section{Conclusions}
The ANN massively outperformed Linear regression in finding an accurate functional relationship between the input and output variables. Classical regression settings may fail to account for multi-collinearity, where variables appear to have large magnitude but opposite side sign coefficients with regard to predicting the response.

\section{Scope for Future Work}
Improvement in results can be made by taking into account factors like Building occupancy and climate conditions data. For eg. usually
larger cooling load occurs between April and September.
Furthermore, the effects of occupant’s presence on a building’s energy consumption vary: people give off heat and pollutants that add to the building’s internal gains and then require more cooling load.
Hence, consideration should be given to increasing the level of accuracy via looking for the parameters which could not only simulate the occupant presence but also reflect their behaviors in the building.
