\textbf{A. Tsanas, A. Xifara (2012)} \cite{sanas2012xifara} developed a statistical machine learning framework using a range of diverse input variables to study heating load and cooling load requirements of a building. It demonstrated that it is possible to accurately estimate HL with only 0.5 points deviation and CL with 1.5 points deviation from the ground truth (the simulated results). These findings are particularly compelling given the accurate prediction, and also because we can easily infer the output variables in a matter of few seconds without requiring the painstaking design of a new building in a simulation tool.

\textbf{A. Hani and T. Koiv (2012)} \cite{hani2012koiv} analyzes the thermal and electrical energy consumptions for 40 Residential, 7 Educational and 44 Public buildings located in warm summer continental climate. Information about reconstructions, heating source, internal air temperature, air exchange rate and domestic hot water production is tabulated.

\textbf{Dong, B., Cao, C., \& Lee, S. E. (2005)} \cite{dong2005cao} This paper presents support vector machines (SVM), a new neural network algorithm, to forecast building energy consumption in the tropical region. The objective of this paper is to examine the feasibility and applicability of SVM in building load forecasting area. Four commercial buildings in Singapore are selected randomly as case studies. Weather data including monthly mean outdoor dry-bulb temperature (T0), relative humidity (RH) and global solar radiation (GSR) are taken as three input features. Mean monthly landlord utility bills are collected for developing and testing models. Finally, all prediction results are found to have coefficients of variance (CV) less than 3\% and percentage error (\%error) within 4\%.

\textbf{S.S.K. Kwok, R.K.K. Yuen, E.W.M. Lee (2011)} \cite{kwok2011yuen} This paper discusses the use of the multi-layer perceptron (MLP) model, one of the artificial neural network (ANN) models widely adopted in engineering applications, to estimate the cooling load of a building. The training samples used include weather data obtained from the Hong Kong Observatory and building-related data acquired from an existing prestigious commercial building in Hong Kong that houses a mega complex and operates 24 h a day. The paper also discusses the practical difficulties encountered in acquiring building-related data. In contrast to other studies that use ANN models to predict building cooling load, this paper includes the building occupancy rate as one of the input parameters used to determine building cooling load. The results demonstrate that the building occupancy rate plays a critical role in building cooling load prediction and significantly improves predictive accuracy.

\textbf{Yu, Zhun and Haghighat, Fariborz and Fung, Benjamin C.M. and Yoshino, Hiroshi (2010)} \cite{yu2010zhun} This paper reports the development of a building energy demand predictive model based on the decision tree method. The developed model estimates the building energy performance indexes in a rapid and easy way. This method is appropriate to classify and predict categorical variables: its competitive advantage over other widely used modeling techniques, such as regression method and ANN method, lies in the ability to generate accurate predictive models with interpretable flowchart-like tree structures that enable users to quickly extract useful information.
