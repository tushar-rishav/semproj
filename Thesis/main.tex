\documentclass[a4paper,12pt,oneside]{book}
\usepackage[utf8]{inputenc}
\usepackage{amssymb}
\usepackage{amsmath}
\usepackage{graphicx}
\graphicspath{ {images/} }
\usepackage{times}
\usepackage{geometry}
\usepackage{setspace}
\usepackage{tocloft}
\usepackage{tabu}
\usepackage{fancyhdr}
\usepackage{tabularx}
\usepackage{caption}
\usepackage{listings}  % for highlighting source code.
\usepackage{color}
  \lstset{ % General setup for the package
	  language=Python,
  	basicstyle=\small\sffamily,
	  numbers=left,
  	numberstyle=\tiny,
  	frame=tb,
	  tabsize=2,
	  columns=fixed,
	  showstringspaces=false,
	  showtabs=false,
	  keepspaces,
	  commentstyle=\color{red},
	  keywordstyle=\color{blue}
  }

%Defining some new commands for some repeated text.
%Modify your name and other details in this section.
\newcommand{\theauthor}{Tushar Gautam}
\newcommand{\therollno}{114113089}
\newcommand{\thedegree}{B.Tech}
\newcommand{\thedegreelong}{Bachelor of Technology}
\newcommand{\thetitle}{Intelligent estimation of heating and cooling
load requirements of buildings}
\newcommand{\theguide}{Dr.K.Panneerselvam}
\newcommand{\thecoguide}{Dr.E.S.Gopi}
\newcommand{\thehod}{Dr.M.Duraiselvam}
\newcommand{\thedepartment}{Production Engineering}
\newcommand{\theyear}{2017}
\newcommand{\theacadyear}{2016-2017}
\newcommand{\themonth}{April}
\newcommand{\E}{\mathrm{E}}
\newcommand{\Var}{\mathrm{Var}}
\newcommand{\Cov}{\mathrm{Cov}}
\newcommand{\Corr}{\mathrm{Corr}}



%Adjusting the caption settings
\captionsetup[figure]{labelfont=bf, textfont=bf}
\captionsetup[table]{labelfont=bf, textfont=bf, position=top}

%Set paragraph indentation to zero
\setlength\parindent{0pt}
%Set paragraph spacing
\setlength{\parskip}{12pt}

%Set the paper size and the margins
\geometry{a4paper, tmargin=1in, rmargin=1in, bmargin=1in, lmargin=1.5in}
\usepackage{titlesec}
\usepackage[hidelinks]{hyperref}

%Format the chapter headings to be 14pt, centered and Uppercase
\titleformat{\chapter}[display]
  {\large\bfseries\centering}
  {\MakeUppercase\chaptertitlename\ \thechapter}{7pt}{\large\MakeUppercase}

%Format the section headings to be uppercase and 12pt
\titleformat{\section}[hang]
  {\MakeUppercase\normalfont\bfseries}
  {\thesection}{12pt}{\MakeUppercase}

%Format the subsections to be 12pt
\titleformat{\subsection}[hang]
  {\MakeUppercase\normalfont\bfseries}
  {\thesubsection}{12pt}{}

%Set the spacing between the chapter headings and the margins
\titlespacing*{\chapter}{0pt}{0pt}{24pt}
\titlespacing*{\section}{0pt}{0pt}{0pt}
\titlespacing*{\subsection}{0pt}{0pt}{0pt}
%Add dots to the table of contents for chapters
\renewcommand{\cftchapleader}{\cftdotfill{\cftdotsep}} % for chapters
%Set the spacings before and after the titles for the TOC, LOF and LOT
\setlength{\cftbeforetoctitleskip}{0.43in}
\setlength{\cftaftertoctitleskip}{21pt}
\setlength{\cftbeforelottitleskip}{0.43in}
\setlength{\cftafterlottitleskip}{21pt}
\setlength{\cftbeforeloftitleskip}{0.43in}
\setlength{\cftafterloftitleskip}{21pt}
%Set the fontsize and formatting for the TOC, LOF and LOT 
\renewcommand\contentsname{\centerline{\fontsize{16pt}{16pt}\selectfont TABLE OF CONTENTS}}
\renewcommand\listfigurename{\centerline{\fontsize{16pt}{16pt}\selectfont LIST OF FIGURES}}
\renewcommand\listtablename{\centerline{\fontsize{16pt}{16pt}\selectfont LIST OF TABLES}}
%Setting page numbers to bottom center
\pagestyle{fancy}
\cfoot{\thepage}
\rhead{}
\lhead{}
\renewcommand{\headrulewidth}{0pt}
\renewcommand{\footrulewidth}{0pt}
%Setting bibliography title format
\renewcommand{\bibname}{\fontsize{16pt}{24pt}\selectfont References \hfill}


\begin{document}
%Adding all the stuff that should be numbered with roman numerals
\frontmatter
\addtocontents{toc}{\textbf{Title}\hfill\textbf{Page No.}\par}
\addtocontents{toc}{\vspace{-0.3cm}}
\input{titlepage}
\input{bonafide}
\addcontentsline{toc}{chapter}{ABSTRACT}
\addtocontents{toc}{\vspace{-0.3cm}}
\thispagestyle{plain}
\begin{center}
\textbf{\textbf{\fontsize{16pt}{24pt}\selectfont ABSTRACT}}
\end{center}

\vspace{0.3cm}
\fontsize{12pt}{18pt}\selectfont This project studies the effect of various input variables
(relative compactness, surface area, wall area, roof area, overall height, orientation, glazing area, glazing area distribution) on two output variables, namely heating load (HL) and cooling load (CL) of buildings. Reports suggest that building energy consumption has
steadily increased over the past decades worldwide \cite{perez2008pout}, \cite{cai2009ren}, and heating, ventilation and air conditioning
(HVAC), which have a catalytic role in regulating the indoor climate, account for most of the energy use in
the buildings. Therefore, one way to alleviate the ever increasing demand for additional energy supply is to
have more energy-efficient building designs with improved energy conservation properties. When it comes
to efficient building design, the computation of the heating load (HL) and the cooling load (CL) is required to
determine the specifications of the heating and cooling equipment needed to maintain comfortable indoor air
conditions.
Various Machine Learning approach (Regression, SVM, ANN etc) is to be studied, implemented and
compared on a datasets obtained from Machine Learning Repository, Centre for Machine Learning and
Intelligent Systems, UCI. Finally, a model with the best accuracy is proposed.

\textit{Keywords}: Building energy evaluation; Heating load; Cooling load; Regression; SVM; ANN

\newpage

\addcontentsline{toc}{chapter}{ACKNOWLEDGEMENTS}
\addtocontents{toc}{\vspace{-0.3cm}}
\thispagestyle{plain}
\begin{center}
\textbf{\textbf{\fontsize{16pt}{24pt}\selectfont ACKNOWLEDGEMENTS}}
\end{center}

I wish to express my sincere thanks to Dr.K.Pannerselvam, for reviewing my work and constant support throughout the project.
I place on record, my sincere thank you to the Dr.M.Duraiselvam and Dr.S.Vinodh for their continuos encouragement to pursue the project in the first place.

I am also grateful to Dr. E.S.Gopi, Assistant Professor, in the Department of Electronics and Communication Engineering. I am extremly thankful and indebted to him for sharing expertise, sincere and valuable guidance and encouragement extended to me. Furthermore, taking a course on Pattern Recognition (EC459) by Dr. E.S. Gopi helped significantly.

I also place on record, my sense of gratitude to one and all, who directly or indirectly, have lent their hand in this venture.

\newpage

\addcontentsline{toc}{chapter}{TABLE OF CONTENTS}
\addtocontents{toc}{\vspace{-0.3cm}}
\tableofcontents
\newpage
\addcontentsline{toc}{chapter}{LIST OF TABLES}
\addtocontents{toc}{\vspace{-0.3cm}}
\listoftables
\newpage
\addcontentsline{toc}{chapter}{LIST OF FIGURES}
\addtocontents{toc}{\vspace{-0.3cm}}
\listoffigures
 
\mainmatter

\chapter{Introduction}
\fontsize{12pt}{18pt}\selectfont
\input{Chapter1}

\chapter{Review of Literature}
\fontsize{12pt}{18pt}\selectfont
\textbf{A. Tsanas, A. Xifara (2012)} \cite{sanas2012xifara} developed a statistical machine learning framework using a range of diverse input variables to study heating load and cooling load requirements of a building. It demonstrated that it is possible to accurately estimate HL with only 0.5 points deviation and CL with 1.5 points deviation from the ground truth (the simulated results). These findings are particularly compelling given the accurate prediction, and also because we can easily infer the output variables in a matter of few seconds without requiring the painstaking design of a new building in a simulation tool.

\textbf{A. Hani and T. Koiv (2012)} \cite{hani2012koiv} analyzes the thermal and electrical energy consumptions for 40 Residential, 7 Educational and 44 Public buildings located in warm summer continental climate. Information about reconstructions, heating source, internal air temperature, air exchange rate and domestic hot water production is tabulated.

\textbf{Dong, B., Cao, C., \& Lee, S. E. (2005)} \cite{dong2005cao} This paper presents support vector machines (SVM), a new neural network algorithm, to forecast building energy consumption in the tropical region. The objective of this paper is to examine the feasibility and applicability of SVM in building load forecasting area. Four commercial buildings in Singapore are selected randomly as case studies. Weather data including monthly mean outdoor dry-bulb temperature (T0), relative humidity (RH) and global solar radiation (GSR) are taken as three input features. Mean monthly landlord utility bills are collected for developing and testing models. Finally, all prediction results are found to have coefficients of variance (CV) less than 3\% and percentage error (\%error) within 4\%.

\textbf{S.S.K. Kwok, R.K.K. Yuen, E.W.M. Lee (2011)} \cite{kwok2011yuen} This paper discusses the use of the multi-layer perceptron (MLP) model, one of the artificial neural network (ANN) models widely adopted in engineering applications, to estimate the cooling load of a building. The training samples used include weather data obtained from the Hong Kong Observatory and building-related data acquired from an existing prestigious commercial building in Hong Kong that houses a mega complex and operates 24 h a day. The paper also discusses the practical difficulties encountered in acquiring building-related data. In contrast to other studies that use ANN models to predict building cooling load, this paper includes the building occupancy rate as one of the input parameters used to determine building cooling load. The results demonstrate that the building occupancy rate plays a critical role in building cooling load prediction and significantly improves predictive accuracy.

\textbf{Yu, Zhun and Haghighat, Fariborz and Fung, Benjamin C.M. and Yoshino, Hiroshi (2010)} \cite{yu2010zhun} This paper reports the development of a building energy demand predictive model based on the decision tree method. The developed model estimates the building energy performance indexes in a rapid and easy way. This method is appropriate to classify and predict categorical variables: its competitive advantage over other widely used modeling techniques, such as regression method and ANN method, lies in the ability to generate accurate predictive models with interpretable flowchart-like tree structures that enable users to quickly extract useful information.


\chapter{Data exploration}
\fontsize{12pt}{18pt}\selectfont
\section{Dataset description}
  The dataset contains eight attributes (or features, denoted by X1...X8) and two responses (or outcomes, denoted by y1 and y2).
  \begin{table}[h!]
  \parbox{.45\linewidth}{
          \centering
          \caption{Features}
          \label{tab:table1}
          \begin{tabular}{|c|}
            Relative compactness (X1)\\
            \hline
            Surface area (X2)\\
            \hline
            Wall area (X3)\\
            \hline
            Roof area (X4)\\
            \hline
            Overall height (X5)\\
            \hline
            Orientation (X6)\\
            \hline
            Glazing area (X7)\\
            \hline
            Glazing area distribution (X8)\\
            \hline            
          \end{tabular}
  }
  \hfill
  \parbox{.45\linewidth}{
          \centering
          \caption{Responses}
          \label{tab:table2}
          \begin{tabular}{|c|}
            Cooling load (y1)\\
            \hline
            Heating load (y2)\\
            \hline           
          \end{tabular}
  }
  \end{table}
  
  \subsection{Mapping}
    The aim is to use the eight features to predict each of the two responses.
    \begin{equation}
         f\colon \begin{array}{>{\displaystyle}r @{} >{{}}c<{{}} @{} >{\displaystyle}l} 
          X &\rightarrow& Y \\ 
          X &\mapsto& f(X) 
         \end{array}
    \end{equation}
        
  \section{Probability density}
  The first step in most data analysis applications is the exploration of the statistical properties of the variables. This is typically achieved   by plotting the probability densities, which succinctly summarize each variable for visualization. One way to obtain an empirical non-parametric density estimate is by using histograms. Although histograms are considered crude for most advanced statistical applications, they have the great advantage of making no prior assumptions regarding the distribution of the examined variable and are very simple to compute. \textit{Often, this preliminary step can reveal whether the variable follows a Gaussian (normal) distribution, which is characterized by a unimodal peak in the middle of the variable’s possible range of values, is completely symmetric}, and is particularly useful because a large number of mathematical functions are applicable \cite{bishop}.

\subsection{Histograms}

\begin{figure}[h!]
    \centering
    \begin{minipage}{0.45\textwidth}
      \centering
      \includegraphics[width=1\textwidth]{hist_cl}
      \caption{Cooling load (y1)}
      \label{fig:hist_cl}
    \end{minipage}\hfill
    \begin{minipage}{0.45\textwidth}
      \centering
      \includegraphics[width=1\textwidth]{hist_hl}
      \caption{Heating load (y2)}
      \label{fig:hist_hl}
    \end{minipage}
\end{figure}
\begin{figure}[h!]
    \centering
    \begin{minipage}{0.45\textwidth}
      \centering
      \includegraphics[width=1\textwidth]{hist_rc}
      \caption{Rel. compactness (X1)}
      \label{fig:hist_rc}
    \end{minipage}\hfill
    \begin{minipage}{0.45\textwidth}
      \centering
      \includegraphics[width=1\textwidth]{hist_sa}
      \caption{Surface area (X2)}
      \label{fig:hist_sa}
    \end{minipage}
\end{figure}
\begin{figure}[h!]
    \centering
    \begin{minipage}{0.45\textwidth}
      \centering
      \includegraphics[width=1\textwidth]{hist_wa}
      \caption{Wall area (X3)}
      \label{fig:hist_wa}
    \end{minipage}\hfill
    \begin{minipage}{0.45\textwidth}
      \centering
      \includegraphics[width=1\textwidth]{hist_ra}
      \caption{Roof area (X4)}
      \label{fig:hist_ra}
    \end{minipage}
\end{figure}
\begin{figure}[h!]
    \centering
    \begin{minipage}{0.45\textwidth}
      \centering
      \includegraphics[width=1\textwidth]{hist_oh}
      \caption{Overall height (X5)}
      \label{fig:hist_oh}
    \end{minipage}\hfill
    \begin{minipage}{0.45\textwidth}
      \centering
      \includegraphics[width=1\textwidth]{hist_or}
      \caption{Orientation (X6)}
      \label{fig:hist_or}
    \end{minipage}
\end{figure}
\begin{figure}[h!]
    \centering
    \begin{minipage}{0.45\textwidth}
      \centering
      \includegraphics[width=1\textwidth]{hist_ga}
      \caption{Glazing area (X7)}
      \label{fig:hist_ga}
    \end{minipage}\hfill
    \begin{minipage}{0.45\textwidth}
      \centering
      \includegraphics[width=1\textwidth]{hist_gad}
      \caption{Glazing area distribution (X8)}
      \label{fig:hist_gad}
    \end{minipage}
\end{figure}
\newpage
\subsection{Conclusion}
We observe that there is no unimodal peak in the middle or the symmetry. Hence, the data is non-Gaussian.

\section{Correlation}
  \subsection{Scatter plot}
    Scatter plots are similar to line graphs in that they use horizontal and vertical axes to plot data points. However, they have a very specific purpose. Scatter plots show how much one variable is affected by another. The relationship between two variables is called their correlation. For simplicity, scatter plots often use normalized data (i.e. all the variables are normalized to lie between 0 and 1) to facilitate comparison between measures that possibly span orders of magnitude different ranges of values.
Correlation plot between different features and between features and responses have been shown below:    
    \begin{figure}
      \centering
      \includegraphics[width=1\textwidth]{scatter_xx}
      \caption{Scatter plot grid between features (X)}
      \label{fig:scatter_xx}
    \end{figure}
    \begin{figure}
      \centering
      \includegraphics[width=1\textwidth]{scatter_xy}
      \caption{Scatter plot grid between features (X) vs responses (Y)}
      \label{fig:scatter_xy}
    \end{figure}
  \subsection{Spearman rank correlation coefficient}
    As concluded from the histogram plot above, the data is non-Gaussian, so we use the Spearman rank correlation coefficient to obtain a statistical metric regarding the association strength of each input variable with each of the two outputs. The Spearman rank correlation coefficient can characterize general monotonic relationships and lies in the range [-1 1], where negative sign indicates inversely proportional and positive sign indicates proportional rela- tionship, whilst the magnitude denotes how strong this relationship is.
    \begin{figure}[htbp]
      \hspace*{-2.1cm}
      \includegraphics[width=1.3\textwidth]{sp_coeff}
      \caption{Spearman rank correlation coeff. for various features}
      \label{fig:sp_coeff}
    \end{figure}
    \newpage
    \begin{table}[h!]
          \centering
          \caption{Association strength estimated using the Spearman rank correlation coefficient of the eight input variables (X1. . . X8) with heating load (y1).}
          \label{tab:sprcoeffhl}
          \begin{tabular}{c|c}
            Features & Spearman rank correlation coefficient\\
            \hline
            X1 & 0.623 \\
            \hline
            X2 & -0.623 \\
            \hline
            X3 & 0.471 \\
            \hline
            X4 & -0.806 \\
            \hline
            X5 & 0.861 \\
            \hline
            X6 & -0.004 \\
            \hline
            X7 & 0.323 \\
            \hline
            X8 & 0.068 \\
            \hline
          \end{tabular}
    \end{table}
    
    \begin{table}[h!]
          \centering
          \caption{Association strength estimated using the Spearman rank correlation coefficient of the eight input variables (X1. . . X8) with cooling load (y2).}
          \label{tab:sprcoeffcl}
          \begin{tabular}{c|c}
            Features & Spearman rank correlation coefficient\\
            \hline
            X1 & 0.651 \\
            \hline
            X2 & -0.651 \\
            \hline
            X3 & 0.416 \\
            \hline
            X4 & -0.803 \\
            \hline
            X5 & 0.865 \\
            \hline
            X6 & 0.018 \\
            \hline
            X7 & 0.289 \\
            \hline
            X8 & 0.046 \\
            \hline
          \end{tabular}
    \end{table}
    \subsection{Correlation Matrix for features(X)}
          \begin{equation}
            \Corr(X, Y) = \frac{\Cov(X, Y)}{\sigma_{x}\sigma_{y}} = \frac{\E[(X-\mu_{x})(Y-\mu_{y})]}{\sigma_{x}\sigma_{y}}  
          \end{equation}
          \newline
          \begin{equation}
            \left[
              \begin{matrix}
                  1.0 & -1.0 & -0.254 & -0.870 & 0.869 & 0.002 & 0.003 & 0.003\\
-1.0 & 1.0 & 0.254 & 0.870 & -0.869 & -0.002 & -0.004 & -0.003\\
-0.254 & 0.254 & 1.0 & -0.195 & 0.221 & -0.001 & -0.001 & -0.001\\
-0.870 & 0.870 & -0.195 & 1.0 & -0.937 & -0.003 & -0.003 & -0.003\\
0.869 & -0.869 & 0.221 & -0.937 & 1.0 & 0.002 & 0.002 & 0.002\\
0.002 & -0.002 & -0.001 & -0.003 & 0.002 & 1.0 & -0.003 & -0.003\\
0.003 & -0.004 & -0.002 & -0.003 & 0.002 & -0.003 & 1.0 & 0.184\\
0.003 & -0.003 & -0.001 & -0.003 & 0.002 & -0.003 & 0.184 & 1.0
              \end{matrix}
            \right]
    \end{equation}
    \begin{figure}[htbp]
      \hspace*{-6.1cm}
      \includegraphics[width=1.7\textwidth]{cov_matrix}
      \caption{Correlation matrix shown as Color Map}
      \label{fig:cov_matrix}
    \end{figure}
    \newpage  
  \subsection{Conclusion}
  Figure \ref{fig:hist_rc} presents the empirical probability distributions of all the input and output variables. These distributions demonstrate that none of the variables follows the normal distribution. Figure \ref{fig:scatter_xy} displays the scatter plots for each of the (normalized) input variables with each of the two output variables. These scatter plots and Color Map show that any functional relationship of the input variables and the output variables is not trivial. This suggests that we can reasonably expect that classical learners such as linear regression (linear models) may fail to find an accurate mapping of the input variables to the output variables. Therefore, these plots intuitively justify the need to experiment with complicated learners such as Artificial Neural Networks (non linear models).


\chapter{Machine learning approach}
\fontsize{12pt}{18pt}\selectfont
\input{Chapter4}

\chapter{Code Snippets}
\fontsize{12pt}{18pt}\selectfont
Below is the implementation in Python programming language. Suitable cross validation (CV), a standard statistical resampling technique has been used. Specifically, the dataset is split into a training subset with which the learner is trained, and a testing subset which is used to assess the learner’s generalization performance. Typically some percentage of the data is left out for testing the learner, and this is known as K-fold CV, where K is usually 5 or 10. In this study, I've used 5-fold CV. The model parameters are derived using the training subset, and errors are computed using the testing subset (out-of-sample error or testing error).
For statistical confidence, the training and testing process is repeated 100 times with the dataset randomly permuted in each run prior to splitting in training and testing subsets.
\newline
\lstinputlisting{main.py}

\chapter{Results and Discussions}
\fontsize{12pt}{18pt}\selectfont
\section{Model accuracy}

  \subsection{Cross validation}
    One of the main reasons for using cross-validation instead of using the conventional validation (e.g. partitioning the data set into two sets of 70\% for training and 30\% for test) is that there is not enough data available to partition it into separate training and test sets without losing significant modelling or testing capability. In these cases, a fair way to properly estimate model prediction performance is to use cross-validation as a powerful general technique. In summary, cross-validation combines (averages) measures of fit (prediction error) to derive a more accurate estimate of model prediction performance.
    \begin{figure}[htbp]
      \hspace*{-2cm}
      \includegraphics[width=1.2\linewidth]{cv.png}
      \caption{Cross-validation score for Ridge regression}
      \label{fig:cv}                    
    \end{figure}
\newpage
  \subsection{Linear vs Non-linear model}
  As observed before, the data has non-linear relationship with responses. The probability distributions demonstrate that none of the variables follows the normal distribution. Furthermore, the scatter plots show that any functional relationship of the input variables and the output variables is not trivial. This suggests that we can reasonably expect that classical learners such as linear regression (linear models) may fail to find an accurate mapping of the input variables to the output variables. Therefore, these insights intuitively show the need for non-linear models. This claim can be justified by comparing the accuracy of a linear model (Ridge regression) and non-linear model (ANN) below:
  \begin{figure}[h!]
      \centering
      \includegraphics[width=0.9\linewidth]{accuracy.png}
      \caption{Models accuracy : Linear (Ridge) vs Non-linear (ANN)}
      \label{fig:accuracy}                    
  \end{figure}



\chapter{Summary and Conclusions}
\fontsize{12pt}{18pt}\selectfont
\section{Summary}
This project investigates the modeling of building heating and cooling load through an intelligent approach that takes into account the
Relative Compactness, Surface Area, Wall Area, Roof Area, Overall Height, Orientation, Glazing Area and Glazing Area Distribution. These findings are particularly compelling given the accurate prediction, and also because we can easily infer the output variables in a matter of few seconds
without requiring the painstaking design of a new building in a simulation tool. 
Further exploration of the statistical relationship between eight input variables and the two output variables was done.

\section{Conclusions}
The ANN massively outperformed Linear regression in finding an accurate functional relationship between the input and output variables. Classical regression settings may fail to account for multi-collinearity, where variables appear to have large magnitude but opposite side sign coefficients with regard to predicting the response.

\section{Scope for Future Work}
Improvement in results can be made by taking into account factors like Building occupancy and climate conditions data. For eg. usually
larger cooling load occurs between April and September.
Furthermore, the effects of occupant’s presence on a building’s energy consumption vary: people give off heat and pollutants that add to the building’s internal gains and then require more cooling load.
Hence, consideration should be given to increasing the level of accuracy via looking for the parameters which could not only simulate the occupant presence but also reflect their behaviors in the building.


\bibliography{tushar_refs}{}
\addcontentsline{toc}{chapter}{REFERENCES}
\bibliographystyle{unsrt}
\end{document}
