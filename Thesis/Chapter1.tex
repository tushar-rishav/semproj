Globally the building sector accounts for more electricity use than any other sector,
42 per cent. No wonder considering that we spend more than 90 per cent of our
time in buildings. With increasing urbanization, higher in developing countries,
the number and size of buildings in urban areas will increase, resulting in an
increased demand for electricity and other forms of energy commonly used in
buildings. Africa’s rate of urbanization of 3.5 per cent per year is the highest in
the world, resulting in more urban areas with bigger populations, as well as the
expansion of existing urban areas. There are currently 40 cities in Africa with
populations of more than a million and it is expected that by 2015 seventy cities
will have populations of one million or more.

In many developing countries there is normally very little margin between existing
power supply and electricity demand. With increasing electricity demand, new
generation needs to be brought in. Although renewable sources of electricity such
as hydro, geothermal or wind provide electricity at a much lower cost, their capital
outlay is large, they are complex and take much longer to implement. Diesel based
generation is usually brought in the short term to meet this demand, which
results in increased cost of electricity.

Investments in energy efficiency in a building can be compared with the cost of
capital investments necessary on the supply side of the energy system to produce
a similar amount of peak capacity or annual energy production. Usually, the
capital costs of efficiency are lower than comparable investments in increased
supply and there are no additional operating costs of efficiency compared to substantial
operating costs for supply-side options. In addition, energy efficiency
investments generally have much shorter lead times than energy supply investments,
a particularly important consideration in countries where the demand for
energy services is growing rapidly.

One consistent quality in the building sector is that it is subject to a high degree
of regulation. Building codes often influence material use and appliance standards
that have a significant effect on energy efficiency. Regulatory regimes, to
the extent that they exist, may therefore provide a pathway to improve efficiency
for both building construction and a variety of building appliances.

Reports \cite{perez2008pout}, \cite{cai2009ren} suggest that heating, ventilation and air conditioning (HVAC), which have a catalytic role in regulating the indoor climate, account for most of the energy use in the buildings. Therefore, one way to alleviate the ever increasing demand for additional energy supply is to have more energy-efficient building designs with improved energy conservation properties. When it comes to efficient building design, the computation of the heating load (HL) and the cooling load (CL) is required to determine the specifications of the heating and cooling equipment needed to maintain comfortable indoor air conditions.

\section {Simulation tools approach}
Building energy simulation tools are currently widely used to analyze or forecast building energy consumption in order to facilitate the design and operation of energy efficient buildings. Simulation tools are used extensively across diverse disciplines because they enable experimentation with parameters that would otherwise be infeasible, or at least very difficult to control in practice \cite{97d2aa88078644a3ab2ef0ca5fd99212}

\section{Machine learning approach}
Using advanced dedicated building energy simulation software may provide reliable solutions to estimate the impact of building design alternatives; however this process can be very time-consuming and requires user expertise in a particular program. Moreover, the accuracy of the estimated results may vary across different building simulation software. Hence, in practice many researchers rely on machine learning tools to study the effect of various building parameters (e.g. compactness) on some variables of interest (e.g. energy) because this is easier and faster if a database of the required ranges of variables is available. Using statistical and machine learning concepts has the distinct advantage that distilled expertise from other disciplines is brought in the EPB domain, and by using these techniques it is extremely fast to obtain answers by varying some building design parameters once a model has been adequately trained. Moreover, statisticalanalysis can enhance our understanding offering quantitative expressions of the factors that affect the quantity (or quantities) of interest that the building designer or architect may wish to focus on.

In this report, various suitable machine learning models (linear and non-linear) have been explored to predict HL and CL.
